% Document is compatible with XeLaTex only. Don't use PDFLaTex!

\documentclass[a4paper,10pt,ngerman]{scrartcl}
\usepackage[ngerman]{babel}
\usepackage{fontspec}
\usepackage[a4paper,margin=2.5cm,footskip=0.5cm]{geometry}
\usepackage{graphicx}
\usepackage{float} %For big H

%Für Algorithmen
\usepackage{algorithm}
\usepackage{algpseudocode}
\usepackage{algorithmicx}
 \usepackage{amsmath}
\algrenewcommand{\Return}{\State\algorithmicreturn~} %for return in new line

% Die nächsten vier Felder bitte anpassen:
\newcommand{\Aufgabe}{Junioraufgabe 2: Hopsitexte}      % Aufgabennummer und Aufgabennamen angeben
\newcommand{\TeamId}{00578}                       % Team-ID aus dem PMS angeben
\newcommand{\TeamName}{Frederik Hamann}           % Team-Namen angeben
\newcommand{\Namen}{Frederik Hamann}              % Namen der Bearbeiter/-innen dieser Aufgabe angeben

% Kopf- und Fußzeilen
\usepackage{scrlayer-scrpage}
\usepackage{lastpage}
\setkomafont{pageheadfoot}{\large\textrm}
\lohead{\Aufgabe}
\rohead{Team-ID: \TeamId}
\cfoot*{\thepage{}/\pageref{LastPage}}

% Position des Titels
\usepackage{titling}
\setlength{\droptitle}{-1.0cm}
% Für Python
\usepackage{pythonhighlight}

%Für PDF-Metadaten
\usepackage{hyperref}
\hypersetup{
pdftitle={Aufgabe 1 - Hopsitexte},
pdfsubject={Hopsitexte},
pdfauthor={Frederik Hamann}
}
\usepackage{cleveref}

% Diese beiden Pakete müssen zuletzt geladen werden
%\usepackage{hyperref} % Anklickbare Links im Dokument

% Daten für die Titelseite
\title{\textbf{\Huge\Aufgabe}}
\author{\LARGE Team-ID: \LARGE \TeamId \\\\
	\LARGE Team-Name: \LARGE \TeamName \\\\
	\LARGE Bearbeiter/-innen dieser Aufgabe: \\ 
	\LARGE \Namen\\\\}
\date{\LARGE\today}

\begin{document}

\maketitle
\tableofcontents

\vspace{0.5cm}


\section{Lösungsidee}
Tatsächlich habe ich erst, nachdem ich die Aufgabe 1 aus dem Aufgabenblatt bearbeitet habe, festgestellt, dass ich die Junioraufgaben auch bearbeiten darf, daher habe ich diese Aufgabe erst nach dem Bearbeiten von Aufgabe 1 gelöst, wodruch ich im Prinzip das Programm aus Aufabe 1 für die Aufgabe wieder verwendet habe und nur in Teilaspekten gekürzt habe und den Algorithmus welcher in Aufgabe 1 von mir verwendet wird nun so nutze, dass er anstatt den Abstand von Hopser 1 und 2 zu berechnen, er untersucht ob Hopser 1 oder 2 weiter gekommen ist.


\begin{algorithm}
\caption{check\_win()}
\begin{algorithmic}
    \State \textbf{global} winner
    \While{\textbf{true}}
        \State check\_hopsi(hopser\_1)
        \State check\_hopsi(hopser\_2)
        \If{check\_hopsi(0) $>$ check\_hopsi(1)}
            \State winner $\gets$ "Hopser 1 gewinnt!"
        \ElsIf{check\_hopsi(1) $>$ check\_hopsi(0)}
            \State winner $\gets$ "Hopser 2 gewinnt!"
        \Else
            \State winner $\gets$ "Unentschieden!"
        \EndIf
    \EndWhile
\end{algorithmic}
\end{algorithm}

\section{Umsetzung}
Es wird mithilfe der Python-Bibliothek „ttkbootstrap“ ein GUI erstellt, welches ein Textfeld und 
eine Radialanzeige enthält. Das Textfeld dient der Eingabe des Hopsitextes, während die Radialanzeige
zur Darstellung des Abstands zwischen den Endpositionen genutzt wird. Es hat einen Anzeigebereich von 0 bis 
29. Der Anzeigebereich wurde auf 29 berechnet, da der höchste Sprungwert bei 30 liegt und der Abstand somit
nur bei max. 29 liegen kann, da der 2. Hopser einen Buchstaben nach dem 1. Texthopser startet und somit in einem Fall,
 wo auf ein ,,ß'' 29 mal ,,a'' folgen würde, der Abstand 29 betrüge (s. Abbildung 3).\\


\section{Beispiele}
Genügend Beispiele einbinden! Die Beispiele von der BwInf-Webseite sollten hier diskutiert werden, aber auch eigene Beispiele 
sind sehr gut, besonders wenn sie Spezialfälle abdecken. Aber bitte nicht 30 Seiten Programmausgabe hier einfügen!




\section{Quellcode}


\end{document}